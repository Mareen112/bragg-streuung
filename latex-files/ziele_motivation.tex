\section{Ziele und Motivation}

Im folgenden Versuch soll das Phänomen der Bragg-Streuung an Neutronen untersucht werden. Im Falle von Röntgenstrahlung stellt sie eine wohl bekannte Beobachtung dar. Aufgrund des Welle-Teilchen-Dualismuses kann dieser Effekt auch mit "klassischen" Teilchen wie Elektronen beobachtet werden.
Dass ein solches Phänomen jedoch auch mit Objekten, die noch aus mehreren Teilchenarten zusammengesetzt sind, auch möglich ist, scheint zunächst schwer vorstellbar. Dass es funktioniert, wollen wir mit dem folgenden Versuch zeigen. Dabei kommt auch wieder eindrucksvoll der Welle-Teilchen-Dualismus zum Tragen, der offensichtlich auch für Neutronen funktioniert. Eine Streuung an Kristallatomen ist nur im Wellenbild erklärbar, während der Nachweis der Neutronen über die Absorption durch Zählgasatome nur über eine Beschreibung des Neutrons als Teilchen erklärbar ist. 

\subsubsection{Geschichte}

1932 entdeckte J. Chadwick das von E. Rutherford bereits 1920 vorausgesagte Neutron. Grundlage für seine Entdeckung war ein Experiment, welches von Walther Bothe und seinem Studenten Herbert Becker durchgeführt wurde, bei dem sie Beryllium mit Alphastrahlung beschossen. Sie hielten die bei der Reaktion entstandene stark durchdringende und energiereiche Strahlung, da sie auch nicht auf äußere elektrische Felder reagierte, fälschlicherweise für Gammastrahlung. Doch als Endprodukt entstand nicht wie vorher erwartet Bor, sondern Kohlenstoff.

Die entdeckte Strahlung war viel energiereicher als jede zuvor entdeckte Gammastrahlung und so kamen Zweifel daran auf, ob es sich wirklich Gammastrahlung handelt. Chadwick, der nicht überzeugt von der Erklärung mit der Gammastrahlung war, entwickelte eine schnell eine Reihe von Experimenten, in denen sich herausstellte, dass es sich bei der Beryllium-Strahlung um ungeladene Teilchen mit der ungefähren Masse von Protonen handeln muss. Für diese Entdeckung erhielt er 1935 den Nobelpreis für Physik.
Aufgrund des Welle-Teilchen-Dualismus der Neutronen und der Entwicklung der Elektronenbeugung, konnte H. Halban 1936 zeigen, dass Neutronenbeugung in der Theorie möglich ist. Der experimentelle Beweis blieb aber noch aus, da zu dieser Zeit die Neutronenquellen noch keine starken Flüsse erzeugen konnten. Mit der Entdeckung der Kernspaltung durch Lise Meitner und Otto Hahn im Jahre 1938 und der darauffolgenden Entwicklung der ersten Kernreaktoren, konnte erstmals 1944 ein wirklich starker Neutronenfluss generiert werden. Seitdem wurden zahlreiche Neutronenbeugungsexperimente durchgeführt. In den ersten paar Jahren konzentrierte man sich auf Beugungsversuche und fand heraus, dass sich inelastische Neutronenstreuung sehr gut zur Spektroskopie eignet. Dank der fehlenden Ladung können die Neutronen tief in Materie eindringen (interagieren nur mit den Kernen) und werden nicht von Oberflächeneffekten beeinflusst, was zum Beispiel ein Nachteil der Röntgenstrahlung ist.
